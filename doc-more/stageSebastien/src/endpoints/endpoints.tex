\documentclass{article}
%packages
\usepackage{graphicx}
\usepackage{hyperref}
\usepackage[utf8]{inputenc}
\usepackage[T1]{fontenc}
\usepackage[a4paper]{geometry}
\usepackage{minted}

\title{endpoints.h}
\date{July 13th 2015}

\begin{document}
\maketitle

\tableofcontents

\section{Description}
In our case, we run a COAP server on the Arduino. This server can send a response to any COAP request which the arduino receives through the LoRa Network. So, we can imagine a light sensor which send its value when we make a GET request. In the following documentation, the name of the Arduino object will be \emph{toto.s.ackl.io}.

\section{Add a point}
\subsection{Basics}
To add a point, we need 2 things. First, we need a path to access this point. We declare the path as following:\\
\begin{minted}{c}
//First create a coap_endpoint_path_t
static const coap_endpoint_path_t our_path = {1, {"example"}};
//1 is the length of the dictionnary ({"example"})

//Then, add this path to the endpoints
const coap_endpoint_t endpoints[] =
{
    {COAP_METHOD_GET, handle_get_well_known_core, &path_well_known_core, "ct=40"},
    {COAP_METHOD_GET, handle_example, &our_path, "ct=0"},
    {(coap_method_t)0, NULL, NULL, NULL}
};
\end{minted}
With this code, we will match the path \emph{toto.s.ackl.io/example} and start handle\_example. We must to create the handle:\\
\begin{minted}{c}
static int handle_example(coap_rw_buffer_t *scratch, const coap_packet_t *inpkt, 
coap_packet_t *outpkt, uint8_t id_hi, uint8_t id_lo)
{
  Serial.print("Example!");
  char lightMsg[] = "Example!";

  return coap_make_response(scratch, outpkt, (const uint8_t *)lightMsg, 
  strlen(lightMsg), id_hi, id_lo, &inpkt->tok, COAP_RSPCODE_CONTENT, 
  COAP_CONTENTTYPE_TEXT_PLAIN);
}
\end{minted}
\subsection{Interact with the Arduino}
The file endpoints.h include the Arduino library. So, you can write Arduino code in the handle without any problem. For example, we can imagine a handle to turn on a led:\\
\begin{minted}{c}
static const coap_endpoint_path_t path_led_on = {1, {"ledon"}};

static int handle_get_led_on(coap_rw_buffer_t *scratch, const coap_packet_t *inpkt,
coap_packet_t *outpkt, uint8_t id_hi, uint8_t id_lo)
{
  Serial.println("led on!");
  digitalWrite(led, HIGH);
  char ledonmsg[] = "the led is on";  
  
  return coap_make_response(scratch, outpkt, (const uint8_t *)ledonmsg, 
  strlen(ledonmsg), id_hi, id_lo, &inpkt->tok, COAP_RSPCODE_CONTENT, 
  COAP_CONTENTTYPE_TEXT_PLAIN);
}
\end{minted}
\subsection{Return types}
With the previous examples, we send a text response. But we can respond other things:
\begin{itemize}
  \item COAP\_CONTENTTYPE\_NONE
  \item COAP\_CONTENTTYPE\_TEXT\_PLAIN
  \item COAP\_CONTENTTYPE\_APPLICATION\_LINKFORMAT
  \item COAP\_CONTENTTYPE\_APPLICATION\_XML
\end{itemize}
\subsection{Example}
To send a webpage:\\
\begin{minted}{c}
static const coap_endpoint_path_t path_helloWorld = {1, {"helloWorld"}};

static int handle_get_helloWorld(coap_rw_buffer_t *scratch, const coap_packet_t *inpkt,
coap_packet_t *outpkt, uint8_t id_hi, uint8_t id_lo)
{
  Serial.println("hello World!");
  
  char hellomsg[] = "<html xmlns=\"http://www.w3.org/1999/xhtml\"><center><h1>
  Hello World :) </h1></center></html>";
  
  return coap_make_response(scratch, outpkt, (const uint8_t *)hellomsg, strlen(hellomsg), 
  id_hi, id_lo, &inpkt->tok, COAP_RSPCODE_CONTENT, COAP_CONTENTTYPE_APPLICATION_XML);
}
\end{minted}
\end{document}

